%Chapter 1: Introduction

\chapter{Introduction}
\label{chap:intro}
\lhead{Chapter 1. \emph{Introduction}}

Begin with observational evidence of AGN. They are extragalactic sources whose main power is not from the stellar nuclear reactions. After that we prove that given the luminosity, time variability and density of energy, the origin is a Supermassive black hole.

After that mention that the AGN are long lived objects, that given its visibility we can detect them at high redshifts, and then are a useful tool to study the evolution of the universe.

AGN are very important to understand the evolution of the galaxies, as there is a coevolution of SMBH and host galaxies via feedback.

After that we may mention the X-ray background, compound by AGNs. In the soft energies the background is explained, but at hard energies is not. The obscured AGN still are something not completely understood. That is why we must study the extinction in AGNs. Extinction plays a major role in AGN subdivision in types that is not following unified models for part of the population.


\section{Unified Model of AGN}
\label{sec1:um}

List all the regions of the unified model and where the emission mechanisms are originated.

-Disk: infalling material that emits in the optical/UV
-Corona: inverse compton scattered emission that is triggered by the accretion disk
-BLR: ionized gas region outside the disk.
-NLR:less ionized material outside de BLR
-Torus:Dusty region surrounding the BLR,disk,corona
-Jets: polar radio emission.


\section{Physical mechanisms of emission}
\label{sec1:agn}
Describe the emission mechanisms of the AGN in each wavelength. This is:

Radio: introducing two types that contributes differently to the bolometric luminosity, but later on we will explain more.
IR: Emission from the heated dust, forming a IR bump.
Optical/UV: THe Big Blue Bump, that also have broad and narrow emission lines. THere are pseudo continuum contribution that are blends of lines and not a real continuum emission:
X-ray: power law emission. Maybe in this part introduce as well all the physical origins of X-ray emission
Gamma-ray: for the non-thermal continuum, of Blazards

We may put as a subsection the extinction?



\section{Classification of AGN}
\label{sec1:class}

Here we describe the classification and subclassification of AGN.

-We first explain the orientation effects in the observed spectra and SED with a figure where it is seen the type depending on the orientation.

-After that we put a table and explain other classifications where there is dependency on other factors as the luminosity (QSO) or FE emission/FWHM (NLS1)

-We introduce as well the subclassification within type-1 linking it with different levels of extinction

If not mentioned before, we focus on classification of type-1 and 2 based on X-ray and optical data.

\section{Absorption and obscuration}
\label{sec1:abs}

In this section we begin to introduce some aspects of importance in the thesis that are the differences between X-ray and optical extinction of AGN. We explain the effects of absorption in the X-rays and the effects of optical obscuration in UV/optical/IR.

-Gas absorption in the X-rays

-Extinction by dust in the UV/optical/IR

We point out that this is not followed by a significant fraction of sources.

-We explain the fraction of discordant sources and possible origin in the literature. Different dust to gas ratio, different absorption regions, ... and variability.


\section{Aims of this thesis}
\label{sec1:aim}

Describe the aims of the papers in this thesis.

Summarizing, along this work we will tackle the following issues:
\begin{enumerate}
\item One

\item Two

\item Three

\end{enumerate}

