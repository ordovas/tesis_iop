%Chapter 3

\chapter{The BUXS Sample} % Write in your own chapter title
\label{chap:buxs}
\lhead{Chapter 3. \emph{The BUXS Sample}} % Write in your own chapter title to set the page header

         
%- BUXS y submuestra de tipos-1
%- Ajustes y extracción rayos-X
%- Ajustes del contínuo óptico/NIR de AGN
% - Ajustes de líneas

            - SEDs
\section{Sample definition}
\label{sec3:samp}
In this section we explain how the BUXS sample is selected and the final selection, optical completeness.

\section{X-ray modeling}
\label{sec3:xray}

In this section we explain how the X-ray data were treated. We explain that we combine all the available observations. After that sources were extracted using info from other part. Apart from that we explain that the spectral modelig were based on XMMFITCAT. This is how we select the best model, how we calculate the parameters as the luminosity and nh and how compute errors.

\section{Optical spectral continuum modeling}
\label{sec3:op}

In this section we explain the reduction of an optical spectrum. We may distinguish between echelle of xshooter and long slit spectra from other telescopes, as one was analyzed with STARLIGHT and power laws, and the other with SHERPA and combinations of templates.

Apart from that we explain the ways of fitting the optical spectra. We talk about STARLIGHT for the xshooter, and sherpa for the BUXS sample.

The general model is AGN plus SMC extinction and additive host galaxy model. For some objects where we have data in the high order f the Balmer lines and higher, it is necessary to introduce FeII and Balmer Continuum emission, so we describe all the components here.


\section{Optical emision lines fits}
\label{sec3:lines}

Here we describe the H$_{\rm \alpha}$, H$_{\rm \beta}$ and MgII line fit models. The NLR uses the same width in velocity. We use the FWHM and flux as free parameters.


\section{Subsample used in this thesis}
\label{sec3:sample}

Here we give details of the data used in this thesis. In the following chapters we explain more precisely the subsamples.