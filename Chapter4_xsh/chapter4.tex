\chapter{Detailed modeling of two sources} % Write in your own chapter title
\label{chap:xsh}

\lhead{Chapter 4. \emph{Detailed modeling of two sources}}

This chapter is about the two discordant AGN observed with XSHOOTER. We re-format the article to show the results and possible origin of the discordance.


\section{Subsample of two objects}
\label{sec4:samp}
We describe here the two objects selected with XSHOOTER



\section{X-ray properties}
\label{sec4:xray}
Here we explain the X-ray spectral fitting of the two sources.


\section{UV-to NIR observations}
\label{sec4:xsh}
In this section we describe the observations from the VLT.



\section{Analysis and results}
\label{sec4:an}

\subsection{AGN and host galaxy continuum decomposition}
\label{sec4:starlight}

With STARLIGHT we decompose the extracted spectrum and divide it into host galaxy and AGN emission, with the STARLIGHT software.

\subsection{Narrow and broad line Balmer decrements}
\label{sec4:lines}

After removing the host galaxy contamination, we fit the emission lines.

\subsection{SMBH masses}
\label{sec4:bhmass}

Using the broad H$_{\alpha}$~line, we obtain the SMBH masses of each AGN.


\subsection{Host galaxy masses}
\label{sec4:galmas}

\subsubsection{Stellar masses}
\label{sec4:stmass}
We obtain the host galaxy stellar mass from the STARLIGHT software

\subsubsection{Dynamical masses}
\label{sec4:dynmas}
From the NaID, we estimate the host galaxy dynamical masses, and compare it with the stellar masses.



\section{Discussion}
\label{sec4:disc}
Here we describe the possible causes of the discordance between the optical and X-ray classifications.

\subsection{Compton-thick or Compton-thin obscuration}
\label{sec4:compton}

Using line flux ratios, we determine that the sources are not Compton-thick AGN.


\subsection{Host dilution}
\label{sec4:dil}

We compare the SMBH and the host galaxy masses to check if the host galaxy is more massive than expected through the SMBH and the host galaxy relations.

\subsection{Dust-to-gas ratio ot the obscuring medium}
\label{sec4:avnh}

We check if the obscuring medium is more dusty than the Galactic dust-to-gas relation.


\subsection{Intrinsically weak BLR region}
\label{sec4:blr}

To check if the broad lines are underluminous and that is why they are hard to detect.


\subsection{Variability}
\label{sec4:var}

We explain the possible impact of variability in the sources.


\section{Results}
\label{sec4:res}

We point out that one source have higher dust-to-gas ratio than the Galactic, and other is hosted by a massive galaxy.



