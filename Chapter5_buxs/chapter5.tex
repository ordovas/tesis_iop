

\chapter{Analysis of BUXS sample}
\label{chap:type1}
\lhead{Chapter 5. \emph{Analysis of BUXS sample}}

We study the obscuration of type-1 AGN by comparing the optical extinction and the X-ray absorption.


\section{Sample definition}
\label{sec5:samp}

We describe here that we select all objects that show at least one broad line in their optical spectrum, that is ranging from type-1.0 to type-1.8/9, often grouped with type-2 AGN. We use only the redshift range of z=0.05-1 to measure in a robust way the X-ray obscuration.

\section{X-ray and optical}

\subsection{X-ray properties}

Here we explain the X-ray spectral fitting of the type-1 sources. In this section we also examine the percentage of X-ray absorbed sources and compare it with other samples. We test the evolution of the X-ray luminosity with the  fraction of absorbed sources.


\subsection{Optical spectrum fits}

We describe the model used to fit the optical spectrum. This allows us to measure the optical extinction of the sources in terms of Av. We can compare here the Av range of other selections and the fraction of sources not optically obscured.

\subsubsection{SED Av vs spectrum Av}

We compare this estimations with the ones from the SED analysis.

\subsubsection{Balmer decrement Av vs spectrum Av}

We compare this estimations with the ones where H$_{\alpha}$ and H$_{\beta}$ are available. We test if there is an intrinsic H$_{\alpha}$/H$_{\beta}$ ratio, or it depends on the conditions on the BLR.



\section{Subdivision in Seyfert subclases}

We study the change in parameters such as Av, NH, etc with the Seyfert subclass. We also check with different redshifts if there is an evolution or not.


\section{Optical extinction versus X-ray absorption}

We plot the Av vs NH, and we compute the fraction of sources that follows the Galactic dust-to-gas relation, the ones that are more dusty and the ones that have more gas.

\section{Dust-to-gas ratio}

Plotting the dust-to-gas ratio versus the luminosity or redshift to test if there is any dependence in between those quantities.



\section{Bolometric luminosity and Bolometric correction}

Here we test if the relations between the Bolometric luminosity based on the optical spectrum and the luminosity of the X-rays is compatible with the ones reported in other studies.



\section{Conclusions of the statistical study}

We summarize the main differences that this complete sample of type-1 AGN have with other optical or X-ray selected samples. We explain this differences with context with the unified model of AGN and with the latest explanations in the literature of optical extinction and X-ray absorption.





 